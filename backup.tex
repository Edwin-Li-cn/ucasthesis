\section{变分蒙特卡罗方法求解车间调度问题}
在上节中,我们已经构建了模型,将车间调度问题映射到了一个量子系统,求解车间调度问题也就等价于求解该量子系统的基态。本节中,我们将采用变分蒙特卡罗方法求解该量子系统的基态,用受限玻尔兹曼机当作试探波函数描述对应系统的量子态。
\subsection{受限玻尔兹曼机作为试探波函数}
定义所有合理解的集合为$\mathscr{Y}=\{\mathcal{x}|\mathcal{x}\text{是合理解}\}$。按照硬约束的要求,我们不允许不合理解的存在,因此需要限制受限玻尔兹曼机中可见变量的取值范围:
\begin{equation}
    \psi_{w}(\mathcal{x})=\begin{cases} \frac{1}{Z}e^{\sum_{j} a_{j} x_{j}} \prod_{i} \cosh \left(b_{i}+\sum_{j} w_{i j} x_{j}\right) & \mathcal{x}\in\mathscr{Y} \\ 0 & \mathcal{x}\notin\mathscr{Y} \end{cases}
\end{equation}

其中$Z=\sum\limits_{\mathcal{x}\in\mathscr{Y}}e^{\sum_{j} a_{j} x_{j}} \prod_{i} \cosh \left(b_{i}+\sum_{j} w_{i j} x_{j}\right)$

目标函数即为式~\eqref{eq:JSP_object_function}。我们可以采用~\ref{sec:update_trial_function_method}节介绍的更新试探波函数参数的方法来逐步优化玻尔兹曼机,使得其不断逼近基态波函数。
\subsection{重要性采样条件}
变分蒙特卡罗方法需要对试探波函数进行采样,按照~\ref{sec:Monte_Carlo_sample_method}节介绍的蒙特卡罗采样方法,为使得马尔科夫链能够收敛到静态分布,我们可以采取满足精细平衡条件的方法。

同时,由定理~\ref{theorem:update_convergence}可知,我们可以采取只翻转关键变量的方法,



\chapter{}
\section{:方向接近}
    \begin{enumerate}
        \item \citet{}:
            \begin{itemize}
                \item 工作内容概述:
                \item 课题相关性:
                \item 核心创新点:
                \item 先进性:
                \item 影响力:引用次数:
                \item 相关代码:
            \end{itemize}
        \item \citet{}:
            \begin{itemize}
                \item 工作内容概述:
                \item 课题相关性:
                \item 核心创新点:
                \item 先进性:
                \item 影响力:引用次数:
                \item 相关代码:
            \end{itemize}
    \end{enumerate}
\section{:方向较远}
\begin{enumerate}
    \item \citet{}:
        \begin{itemize}
            \item 工作内容概述:
            \item 课题相关性:
            \item 核心创新点:
            \item 先进性:
            \item 影响力:引用次数:
            \item 相关代码:
        \end{itemize}
    \item \citet{}:
        \begin{itemize}
            \item 工作内容概述:
            \item 课题相关性:
            \item 核心创新点:
            \item 先进性:
            \item 影响力:引用次数:
            \item 相关代码:
        \end{itemize}
\end{enumerate}