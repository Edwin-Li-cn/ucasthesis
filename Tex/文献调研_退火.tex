\chapter{退火相关文献}
\section{退火相关:方向接近}
\subsection{}
    \begin{enumerate}
        \item \citet{finnila1994quantum}:
            \begin{itemize}
                \item 工作内容概述:最先提出基于扩散蒙特卡罗方法(没有引导波函数)的模拟量子退火寻找最优解,在最多包含19个粒子的莱纳德-琼斯集群(连续模型)上进行了测试。
                \item 课题相关性:基于扩散蒙特卡罗方法的模拟量子退火。
                \item 核心创新点:最先提出基于扩散蒙特卡罗方法(没有引导波函数)的模拟量子退火。
                \item 先进性:最先提出基于扩散蒙特卡罗方法(没有引导波函数)的模拟量子退火。
                \item 影响力:引用次数:669。
                \item 相关代码:
            \end{itemize}
            \item \citet{inack2015simulated}:
            \begin{itemize}
                \item 工作内容概述:
                \item 课题相关性:
                \item 核心创新点:
                \item 先进性:
                \item 影响力:引用次数:18。
                \item 相关代码:
            \end{itemize}
        \item \citet{santoro2006optimization}:
        \begin{itemize}
            \item 工作内容概述:
            \item 课题相关性:
            \item 核心创新点:
            \item 先进性:
            \item 影响力:引用次数:344。
            \item 相关代码:
        \end{itemize}
        \item \citet{stella2007quantum}:
            \begin{itemize}
                \item 工作内容概述:
                \item 课题相关性:
                \item 核心创新点:
                \item 先进性:基于格林函数蒙特卡罗方法的模拟量子退火在没有可靠的引导波函数情况下,要劣于基于路径积分蒙特卡罗方法的模拟量子退火。
                \item 影响力:引用次数:22。
                \item 相关代码:
            \end{itemize}
        \item \citet{jarret2016adiabatic}:
            \begin{itemize}
                \item 工作内容概述:
                \item 课题相关性:SSMC属于PQMC;MAX-k-SAT
                \item 核心创新点:
                \item 先进性:
                \item 影响力:引用次数:42。
                \item 相关代码:
            \end{itemize}
    \end{enumerate}
\section{退火相关:方向较远}
\begin{enumerate}
    \item \citet{kirkpatrick1983optimization}:
        \begin{itemize}
            \item 工作内容概述:
            \item 课题相关性:
            \item 核心创新点:
            \item 先进性:
            \item 影响力:引用次数:53969。
            \item 相关代码:
        \end{itemize}
        \item \citet{kadowaki1998quantum}:
            \begin{itemize}
                \item 工作内容概述:在小系统上对多种横向场伊辛模型的含时薛定谔方程进行了数值求解以获得量子退火结果,与经典的拟退火结果进行对比,最终表明量子退火到达基态的概率高于经典拟退火。
                \item 课题相关性:数值求解小系统含时薛定谔方程以描述量子退火。
                \item 核心创新点:在多种横向场伊辛模型上,数值求解小系统含时薛定谔方程以描述量子退火。
                \item 先进性:量子退火在大多数情况下优于经典拟退火。
                \item 影响力:引用次数:1725。
                \item 相关代码:
            \end{itemize}
        \item \citet{brooke1999quantum}:
            \begin{itemize}
                \item 工作内容概述:实验上对量子退火和经典退火在无序磁铁模型上进行了对比,表明量子退火得到的态更接近于基态。
                \item 课题相关性:实验上实现了量子退火。
                \item 核心创新点:实验上实现了量子退火。
                \item 先进性:在该实验模型上表明量子退火优于经典退火。
                \item 影响力:引用次数:553。
                \item 相关代码:
            \end{itemize}
            \item \citet{lee2000global}:
            \begin{itemize}
                \item 工作内容概述:
                \item 课题相关性:
                \item 核心创新点:
                \item 先进性:
                \item 影响力:引用次数:
                \item 相关代码:
            \end{itemize}
        \item \citet{farhi2000quantum}:
        \begin{itemize}
            \item 工作内容概述:提出基于路径积分蒙特卡罗方法的量子热退火(QTA-PIMC),在\citet{honeycutt1992nature}提出的BLN蛋白质模型上进行了测试,与经典拟退火(SA)进行了比较,表明QTA-PIMC优于SA。
            \item 课题相关性:PIMC实现模拟量子退火。
            \item 核心创新点:提出基于路径积分蒙特卡罗方法的量子热退火(QTA-PIMC)
            \item 先进性:传统的SA是QTA-PIMC的子集,QTA-PIMC在大系统上表现更佳。
            \item 影响力:引用次数:1297。
            \item 相关代码:
        \end{itemize}
        \item \citet{lee2001quantum}:
            \begin{itemize}
                \item 工作内容概述:
                \item 课题相关性:PIMC实现模拟量子退火。
                \item 核心创新点:
                \item 先进性:
                \item 影响力:引用次数:28。
                \item 相关代码:
            \end{itemize}
            \item \citet{brooke2001tunable}:
            \begin{itemize}
                \item 工作内容概述:
                \item 课题相关性:
                \item 核心创新点:
                \item 先进性:
                \item 影响力:引用次数:151。
                \item 相关代码:
            \end{itemize}
        \item \citet{farhi2001quantum}:
        \begin{itemize}
            \item 工作内容概述:在小系统上对随机生成的精确覆盖问题的含时薛定谔方程进行了数值求解以获得量子退火结果,结果表明退火所需时间随着所需比特数增长缓慢,为量子退火机解决NPC问题提供了有利证据。
            \item 课题相关性:数值求解小系统含时薛定谔方程以描述量子退火。
            \item 核心创新点:在随机生成的精确覆盖问题上,数值求解小系统含时薛定谔方程以描述量子退火。
            \item 先进性:绝热量子计算的权威文章,为量子退火机解决NPC问题提供了有利证据,一般涉及到绝热量子计算求解优化问题都会引用。
            \item 影响力:引用次数:2046。
            \item 相关代码:
        \end{itemize}
        \item \citet{farhi2002quantum}:
            \begin{itemize}
                \item 工作内容概述:在具体案例上通过理论分析对比研究了绝热量子演化和经典拟退火寻找最优解的能力。提供了两个量子绝热演化(线性时间)优于经典拟退火(指数级时间)的案例。给出结论经典拟退火难以优化的场景在绝热量子演化下或许可以有效解决。
                \item 课题相关性:量子绝热演化优越性的理论分析。
                \item 核心创新点:在具体案例上通过理论分析对比研究了绝热量子演化和经典拟退火寻找最优解的能力。
                \item 先进性:量子绝热演化优于经典拟退火。
                \item 影响力:引用次数:106。
                \item 相关代码:
            \end{itemize}
            \item \citet{martovnak2002quantum}:
            \begin{itemize}
                \item 工作内容概述:
                \item 课题相关性:基于PIMC的模拟量子退火
                \item 核心创新点:
                \item 先进性:
                \item 影响力:引用次数:157。
                \item 相关代码:
            \end{itemize}
        \item \citet{santoro2002theory}:
            \begin{itemize}
                \item 工作内容概述:在二维随机伊辛模型上对基于路径积分蒙特卡罗(PIMC)的模拟量子退火和经典拟退火进行了比较,结果表明二者剩余能量$\epsilon$均满足$\epsilon_{\text{res}}(\tau)=E_{\text{final}}(\tau)-E_{\text{GS}}\approx\log^{-\xi}(\tau)$,但模拟量子退火的$\xi$大于经典拟退火的,因此更为有效。
                同时针对模拟量子退火过程发生的朗道-泽纳隧穿进行了分析,表明能级间隙连续的模型会对退火产生不利影响。
                \item 课题相关性:基于PIMC的模拟量子退火。量子退火因能级间隙导致的局限性。
                \item 核心创新点:在二维随机伊辛模型上对基于PIMC的模拟量子退火剩余能量和退火时间的关系进行了分析。
                \item 先进性:表明基于PIMC的模拟量子退火优于经典拟退火。
                \item 影响力:引用次数:666。
                \item 相关代码:
            \end{itemize}
            \item \citet{bravyi2006complexity}:
            \begin{itemize}
                \item 工作内容概述:
                \item 课题相关性:
                \item 核心创新点:
                \item 先进性:
                \item 影响力:引用次数:257。
                \item 相关代码:
            \end{itemize}
            \item \citet{jansen2007bounds}:
            \begin{itemize}
                \item 工作内容概述:对绝热量子计算提供了数学理论支持,对能级间隙和演化时间的关系进行了解析分析:演化时间$\tau\approx 1/g_{\text{min}}^2$才可以保证$H(t/\tau)$的绝热演化。
                \item 课题相关性:绝热量子计算的理论基础,最小能级间隙对绝热演化时间的限制。
                \item 核心创新点:对能级间隙和演化时间的关系进行了解析分析。
                \item 先进性:对能级间隙和演化时间的关系进行了解析分析。
                \item 影响力:引用次数:361。
                \item 相关代码:
            \end{itemize}
            \item \citet{aharonov2008adiabatic}:
            \begin{itemize}
                \item 工作内容概述:绝热量子计算等价于基于门的标准量子计算。
                \item 课题相关性:
                \item 核心创新点:
                \item 先进性:
                \item 影响力:引用次数:653。
                \item 相关代码:
            \end{itemize}
            \item \citet{bravyi2010complexity}:
            \begin{itemize}
                \item 工作内容概述:
                \item 课题相关性:
                \item 核心创新点:
                \item 先进性:
                \item 影响力:引用次数:133。
                \item 相关代码:
            \end{itemize}
            \item \citet{altshuler2010anderson}:
            \begin{itemize}
                \item 工作内容概述:
                \item 课题相关性:绝热量子优化因能级间隙导致的局限性。
                \item 核心创新点:
                \item 先进性:
                \item 影响力:引用次数:226。
                \item 相关代码:
            \end{itemize}
            \item \citet{farhi2012performance}:
            \begin{itemize}
                \item 工作内容概述:
                \item 课题相关性:绝热量子算法局限性。
                \item 核心创新点:
                \item 先进性:
                \item 影响力:引用次数:122。
                \item 相关代码:
            \end{itemize}
            \item \citet{elgart2012note}:
            \begin{itemize}
                \item 工作内容概述:
                \item 课题相关性:
                \item 核心创新点:
                \item 先进性:
                \item 影响力:引用次数:52。
                \item 相关代码:
            \end{itemize}
            \item \citet{hastings2013obstructions}:
            \begin{itemize}
                \item 工作内容概述:构造了一类无符号问题的stoquastic哈密顿量模型,基于路径积分蒙特卡罗方法的模拟量子退火在此类问题上受到拓扑阻碍不能有效得到基态(线性小的能量间隙却需要指数增长的演化时间)。
                \item 课题相关性:基于路径积分蒙特卡罗方法的模拟量子退火的局限性。
                \item 核心创新点:构造了基于路径积分蒙特卡罗方法的模拟量子退火的受限模型。
                \item 先进性:指出了路径积分蒙特卡罗的局限性,在该类模型上量子绝热优化可能指数级优于PIMC。
                \item 影响力:引用次数:55。
                \item 相关代码:
            \end{itemize}
            \item \citet{boixo2014evidence}:
            \begin{itemize}
                \item 工作内容概述:
                \item 课题相关性:
                \item 核心创新点:
                \item 先进性:
                \item 影响力:引用次数:749。
                \item 相关代码:
            \end{itemize}
            \item \citet{ronnow2014defining}:
            \begin{itemize}
                \item 工作内容概述:
                \item 课题相关性:
                \item 核心创新点:
                \item 先进性:
                \item 影响力:引用次数:587。
                \item 相关代码:
            \end{itemize}
            \item \citet{smolin2014classical}:
            \begin{itemize}
                \item 工作内容概述:对\citet{boixo2014evidence}的工作质疑。
                \item 课题相关性:
                \item 核心创新点:
                \item 先进性:
                \item 影响力:引用次数:89。
                \item 相关代码:
            \end{itemize}
            \item \citet{king2015benchmarking}:
            \begin{itemize}
                \item 工作内容概述:
                \item 课题相关性:
                \item 核心创新点:
                \item 先进性:
                \item 影响力:引用次数:92。
                \item 相关代码:
            \end{itemize}
            \item \citet{heim2015quantum}:
            \begin{itemize}
                \item 工作内容概述:通过数值研究调解了\citet{santoro2002theory}对基于PIMC的模拟量子退火实现量子加速的期望与\citet{ronnow2014defining}没能探测到量子加速实验。
                \item 课题相关性:
                \item 核心创新点:
                \item 先进性:
                \item 影响力:引用次数:210。
                \item 相关代码:
            \end{itemize}
            \item \citet{jarret2015adiabatic}:
            \begin{itemize}
                \item 工作内容概述:提出了一种没有局域较小值但能级间隙随着变量数指数增长的问题,在该类问题下,量子绝热优化不如传统的梯度下降法有效。
                \item 课题相关性:量子绝热优化的局限性。
                \item 核心创新点:提出了一种没有局域较小值但能级间隙随着变量数指数增长的问题。
                \item 先进性:量子绝热优化在特定问题上(能级间隙随着变量数指数增长,但却没有局域较小值)并不一定比传统的梯度下降法有效。
                \item 影响力:引用次数:9
                \item 相关代码:
            \end{itemize}
            \item \citet{liu2015quantum}:
            \begin{itemize}
                \item 工作内容概述:对\citet{boixo2014evidence}的工作质疑。
                \item 课题相关性:
                \item 核心创新点:
                \item 先进性:
                \item 影响力:引用次数:
                \item 相关代码:
            \end{itemize}
            \item \citet{knysh2016zero}:
            \begin{itemize}
                \item 工作内容概述:
                \item 课题相关性:量子退火的局限性。
                \item 核心创新点:
                \item 先进性:
                \item 影响力:引用次数:
                \item 相关代码:
            \end{itemize}
            \item \citet{denchev2016computational}:
            \begin{itemize}
                \item 工作内容概述:对比研究了D-Wave 2X量子退火机、基于路径积分蒙特卡罗方法的模拟量子退火、经典拟退火在解决优化问题上的速度表现。结果表明,D-Wave量子退火机的速度$10^8$快于经典拟退火(单核处理器),且SA退火时间随着问题规模的增长速率远高于D-Wave量子退火机。
                D-Wave量子退火机的速度同样远快于基于PIMC的SQA(单核处理器),最快甚至达到$10^8$倍,但是二者退火时间随着问题规模的增长速率相当。
                \item 课题相关性:量子退火与基于路径积分蒙特卡罗方法的模拟量子退火对比。
                \item 核心创新点:对比研究了D-Wave 2X量子退火机、基于路径积分蒙特卡罗方法的模拟量子退火、经典拟退火在解决优化问题上的速度表现。
                \item 先进性:速度上D-Wave量子退火机具有明显的优势,同时表明模拟量子退火与量子退火的退火时间随着问题规模的增长速率相当。由于经典算法采用的都是单核处理器,文中的速度倍率仅供参考。
                \item 影响力:引用次数:453。谷歌团队提供的量子退火优越性的权威文章,同时也显示出了模拟量子退火的潜力。
                \item 相关代码:
            \end{itemize}
            \item \citet{brady2016quantum}:
            \begin{itemize}
                \item 工作内容概述:对量子绝热优化和路径积分蒙特卡罗方法的隧穿强度进行了数值分析,表明量子蒙特卡罗方法同样受能级间隙的影响。
                \item 课题相关性:路径积分蒙特卡罗方法的隧穿强度。
                \item 核心创新点:之前\citet{liu2015quantum}的研究表明量子蒙特卡罗方法和量子退火的动力学过程不尽相同,该研究确定了量子蒙特卡罗方法同样受能级间隙的影响。
                \item 先进性:
                \item 影响力:引用次数:32。
                \item 相关代码:
            \end{itemize}
        \item \citet{hibat2021variational}:
            \begin{itemize}
                \item 工作内容概述:
                \item 课题相关性:
                \item 核心创新点:
                \item 先进性:
                \item 影响力:引用次数:17。
                \item 相关代码:
            \end{itemize}
            \item \citet{inack2022neural}:
            \begin{itemize}
                \item 工作内容概述:
                \item 课题相关性:
                \item 核心创新点:
                \item 先进性:
                \item 影响力:引用次数:1。
                \item 相关代码:
            \end{itemize}
\end{enumerate}