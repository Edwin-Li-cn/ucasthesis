\chapter{蒙卡相关文献}
\section{蒙卡相关:PQMC/VMC}
    \begin{enumerate}
        \item \citet{anderson1975random}:
            \begin{itemize}
                \item 工作内容概述:提出了扩散蒙特卡罗方法(DMC)
                \item 课题相关性:DMC
                \item 核心创新点:最先提出了DMC
                \item 先进性:最先提出了DMC
                \item 影响力:引用次数:1188。
                \item 相关代码:
            \end{itemize}
        \item \citet{reynolds1982fixed}:
        \begin{itemize}
            \item 工作内容概述:
            \item 课题相关性:
            \item 核心创新点:
            \item 先进性:
            \item 影响力:引用次数:1283。
            \item 相关代码:
        \end{itemize}
        \item \citet{kirkpatrick1983optimization}:
        \begin{itemize}
            \item 工作内容概述:将蒙特卡罗算法用于模拟退火过程,提出经典拟退火算法(SA),并用于求解组合优化问题,在旅行商问题上进行了测试。
            \item 课题相关性:经典拟退火算法(SA);旅行商问题。
            \item 核心创新点:提出经典拟退火算法(SA),并用于求解组合优化问题。
            \item 先进性:有效避免陷入局部最优解的启发式算法。
            \item 影响力:引用次数:53969。
            \item 相关代码:
        \end{itemize}
        \item \citet{ceperley1986quantum}:
            \begin{itemize}
                \item 工作内容概述:对基于随机游走求解多体薛定谔方程的各种方法进行了概述,其中对扩散蒙特卡罗方法(DMC)作了较为详细的介绍,并对各种量子蒙特卡罗方法在凝聚态物理中的应用进行了说明。
                \item 课题相关性:DMC;格林函数蒙特卡罗方法(GFMC);
                \item 核心创新点:
                \item 先进性:表明格林函数蒙特卡罗方法(GFMC)的先进性,应用广泛。
                \item 影响力:引用次数:419。
                \item 相关代码:
            \end{itemize}
        \item \citet{trivedi1990ground}:
            \begin{itemize}
                \item 工作内容概述:
                \item 课题相关性:格林函数蒙特卡罗方法
                \item 核心创新点:
                \item 先进性:
                \item 影响力:引用次数:309。
                \item 相关代码:
            \end{itemize}
        \item \citet{umrigar1993diffusion}:
            \begin{itemize}
                \item 工作内容概述:
                \item 课题相关性:
                \item 核心创新点:
                \item 先进性:
                \item 影响力:引用次数:659。
                \item 相关代码:
            \end{itemize}
            \item \citet{buonaura1998numerical}:
            \begin{itemize}
                \item 工作内容概述:
                \item 课题相关性:GFMC
                \item 核心创新点:
                \item 先进性:
                \item 影响力:引用次数:227。
                \item 相关代码:
            \end{itemize}
            \item \citet{sorella2000green}:
            \begin{itemize}
                \item 工作内容概述:
                \item 课题相关性:
                \item 核心创新点:
                \item 先进性:
                \item 影响力:引用次数:111。
                \item 相关代码:
            \end{itemize}
        \item \citet{nightingale2001optimization}:
        \begin{itemize}
            \item 工作内容概述:提出了线性方法优化试探波函数,将其用于关联函数蒙特卡罗方法,在最多七个粒子的范德瓦尔斯集群模型上进行了测试,得到了基态和激发态能量。
            \item 课题相关性:线性方法优化试探波函数。
            \item 核心创新点:最先提出线性方法优化试探波函数。
            \item 先进性:
            \item 影响力:引用次数:126。线性方法优化试探波函数。
            \item 相关代码:
        \end{itemize}
        \item \citet{foulkes2001quantum}:
            \begin{itemize}
                \item 工作内容概述:对VMC和固定节点DMC进行了教科书式详细的介绍,并在一些案例上(如:固体和集群的基态和激发态)进行了展示。
                \item 课题相关性:VMC和固定节点DMC教科书式详细的介绍。
                \item 核心创新点:
                \item 先进性:
                \item 影响力:引用次数:2446。
                \item 相关代码:
            \end{itemize}
        \item \citet{casula2005diffusion}:
            \begin{itemize}
                \item 工作内容概述:
                \item 课题相关性:
                \item 核心创新点:
                \item 先进性:
                \item 影响力:引用次数:121。
                \item 相关代码:
            \end{itemize}
        \item \citet{umrigar2005energy}:
            \begin{itemize}
                \item 工作内容概述:提出了一种改进版的牛顿法来优化试探波函数。在$NO_2$和$C_{10}H_{12}$分子上对贾斯特罗试探波函数进行了测试。
                \item 课题相关性:在能量减小的方向更新试探波函数参数的方法论。但仍有对海森矩阵求逆等计算量大的操作。
                \item 核心创新点:改进版的牛顿法来优化试探波函数。
                \item 先进性:优于常规的VMC方法中减小方差和局域能量的方法。
                \item 影响力:引用次数:233。
                \item 相关代码:
            \end{itemize}
            \item \citet{sorella2005wave}:
            \begin{itemize}
                \item 工作内容概述:将\citet{casula2004correlated}中的随机重配法和标准的牛顿法结合,称为SRH。在一维海森堡环和二维t-J模型上进行了测试。
                \item 课题相关性:随机重配法和标准的牛顿法结合优化试探波函数。
                \item 核心创新点:将随机重配法和标准的牛顿法结合,称为SRH
                \item 先进性:优于传统的牛顿法。
                \item 影响力:引用次数:234。
                \item 相关代码:
            \end{itemize}
            \item \citet{schmidt2005green}:
            \begin{itemize}
                \item 工作内容概述:
                \item 课题相关性:
                \item 核心创新点:
                \item 先进性:
                \item 影响力:引用次数:16。
                \item 相关代码:
            \end{itemize}
        \item \citet{scemama2006simple}:
            \begin{itemize}
                \item 工作内容概述:将能量涨落势(EFP)方法进行改进,用于优化试探波函数行列式部分中的参数,在丙酮基态和已三烯$1~^{1}B_u$态上进行了测试。
                \item 课题相关性:微扰法优化试探波函数。
                \item 核心创新点:将能量涨落势(EFP)方法进行改进。
                \item 先进性:传统的能量涨落势(EFP)方法稳定、高效,但是算力要求很高,基于此进行简化。
                \item 影响力:引用次数:27。
                \item 相关代码:
            \end{itemize}
        \item \citet{sorella2007weak}:
            \begin{itemize}
                \item 工作内容概述:将\citet{sorella2001generalized}提出的随机重配法进行了改进,用于优化JAGP试探波函数,继而用于\citet{casula2005diffusion}提出的晶格正规化扩散蒙卡方法(LRDMC)对两个苯分子间弱化学键的研究。
                \item 课题相关性:改进的随机重配法优化试探波函数,结合晶格正规化扩散蒙卡方法
                \item 核心创新点:改进的随机重配法,对两个苯分子间弱化学键的模拟研究结果与实验结果相符
                \item 先进性:改进的随机重配法
                \item 影响力:引用次数:235。
                \item 相关代码:
            \end{itemize}
        \item \citet{umrigar2007alleviation}:
            \begin{itemize}
                \item 工作内容概述:改进了\citet{nightingale2001optimization}的线性优化方法,试探波函数的参数增加了非线性项。在$C_2$分子上进行了测试。
                \item 课题相关性:改进的线性方法优化含非线性参数的试探波函数。
                \item 核心创新点:改进的线性方法优化含非线性参数的试探波函数。
                \item 先进性:比\citet{umrigar2005energy}的牛顿法计算量小,容易实现。优于常规的VMC方法中减小方差和局域能量的方法。
                \item 影响力:引用次数:519。
                \item 相关代码:
            \end{itemize}
        \item \citet{toulouse2007optimization}:
            \begin{itemize}
                \item 工作内容概述:对牛顿法\citep{umrigar2005energy}、线性法\citep{umrigar2007alleviation}、微扰法\citep{scemama2006simple}优化贾斯特罗试探波函数的方式在$C_2$分子进行了测试比较和说明。
                \item 课题相关性:优化试探波函数参数方式的比较。
                \item 核心创新点:
                \item 先进性:牛顿法和线性法比较高效,微扰法算力要求低,优化部分参数低效。
                \item 影响力:引用次数:299。
                \item 相关代码:
            \end{itemize}
            \item \citet{nemec2010diffusion}:
            \begin{itemize}
                \item 工作内容概述:
                \item 课题相关性:
                \item 核心创新点:
                \item 先进性:
                \item 影响力:引用次数:42。
                \item 相关代码:
            \end{itemize}
            \item \citet{boninsegni2012population}:
            \begin{itemize}
                \item 工作内容概述:
                \item 课题相关性:
                \item 核心创新点:
                \item 先进性:
                \item 影响力:引用次数:
                \item 相关代码:54。
            \end{itemize}
            \item \citet{pollet2018stochastic}:
            \begin{itemize}
                \item 工作内容概述:
                \item 课题相关性:
                \item 核心创新点:
                \item 先进性:
                \item 影响力:引用次数:5。
                \item 相关代码:
            \end{itemize}
            \item \citet{inack2018understanding}:
            \begin{itemize}
                \item 工作内容概述:
                \item 课题相关性:量子隧穿,DMC
                \item 核心创新点:
                \item 先进性:
                \item 影响力:引用次数:14。
                \item 相关代码:
            \end{itemize}
            \item \citet{parolini2019tunneling}:
            \begin{itemize}
                \item 工作内容概述:基于\citet{inack2018understanding}的工作,在双势阱问题、铁磁体伊辛链、三叶草模型上继续研究引导波函数对PQMC隧穿几率的影响。
                结果发现玻尔兹曼类型的引导波函数、数值精确求解的基态引导波函数、玻尔兹曼机类型的引导波函数运用在PQMC上都显示出了相同的线性减小的隧穿几率趋势:$\propto \mathcal{O}(1/\Delta)$。
                这与\citet{inack2018understanding}工作中,无引导波函数PQMC的隧穿几率趋势一致。
                \item 课题相关性:PQMC引导波函数隧穿几率。
                \item 核心创新点:对引导波函数在PQMC中隧穿几率的作用进行了实验讨论。
                \item 先进性:表明基于带有准确引导波函数PQMC的模拟量子退火可以有效地预测量子退火机的结果。
                \item 影响力:引用次数:7。
                \item 相关代码:
            \end{itemize}
    \end{enumerate}
    
\section{蒙卡相关:方向较远}
\begin{enumerate}
    \item \citet{bravyi2015simulacion}:
    \begin{itemize}
        \item 工作内容概述:
        \item 课题相关性:
        \item 核心创新点:
        \item 先进性:
        \item 影响力:引用次数:1。
        \item 相关代码:
    \end{itemize}
    \item \citet{isakov2016understanding}:
    \begin{itemize}
        \item 工作内容概述:通过数值实验表明路径积分蒙特卡罗方法和非相干隧穿的跃迁时间随着系统规模指数增加的系数相同$\propto \mathcal{O}(\Delta^2)$($\Delta$是最小能级间隙)。
        \item 课题相关性:路径积分蒙特卡罗方法中跃迁几率。
        \item 核心创新点:尽管路径积分蒙特卡罗方法和非相干隧穿的动力学过程不同,数值实验表明二者的跃迁时间随着系统规模指数增加的系数相同。
        \item 先进性:表明基于PIMC的模拟量子退火可以有效地预测量子退火机的结果。
        \item 影响力:引用次数:94。
        \item 相关代码:
    \end{itemize}
    \item \citet{jiang2017scaling}:
    \begin{itemize}
        \item 工作内容概述:通过理论分析,解析的表明路径积分蒙特卡罗方法的逃脱几率与热辅助量子隧穿几率随着系统规模指数级减小的系数相同。
        \item 课题相关性:路径积分蒙特卡罗方法中跃迁几率。
        \item 核心创新点:解析的分析了描述平衡涨落的PIMC和描述非平衡过程非相干隧穿衰减的量子隧穿效应,表明二者的跃迁几率随着系统规模指数级减小的系数相同。
        \item 先进性:为PIMC描述量子绝热演化提供了有利理论支持。
        \item 影响力:引用次数:51。
        \item 相关代码:
    \end{itemize}
    \item \citet{mazzola2017quantum}:
    \begin{itemize}
        \item 工作内容概述:量子蒙卡模拟连续模型中质子转移反应的量子隧穿效应研究。
        \item 课题相关性:量子隧穿,PIMC
        \item 核心创新点:
        \item 先进性:
        \item 影响力:引用次数:28。
        \item 相关代码:
    \end{itemize}
    \item \citet{bravyi2017polynomial}:
    \begin{itemize}
        \item 工作内容概述:
        \item 课题相关性:
        \item 核心创新点:
        \item 先进性:
        \item 影响力:引用次数:34。
        \item 相关代码:
    \end{itemize}
\end{enumerate}