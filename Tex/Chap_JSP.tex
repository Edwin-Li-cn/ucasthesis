\chapter{车间调度问题}\label{chap:JSP}
\section{车间调度问题的定义}
\subsection{车间调度问题的描述}
\subsection{车间调度问题的析取图表示}

\section{变分蒙特卡罗方法求解车间调度问题}
我们首先采用变分蒙特卡罗方法求解车间调度问题,用受限玻尔兹曼机当作试探波函数描述对应系统的量子态。首先我们需要构建模型,将车间调度问题映射到一个量子系统,求解车间调度问题也就等价于求解该量子系统的基态。
\subsection{模型构建}
设待求解车间调度问题总的工序数量为$N$,对各个工件上的各个工序依次标号$1\sim N$。析取图中,虚拟的开始工序标号$0$,虚拟的结束工序标号$N+1$

对于任意两个不属于同一工件但需要在同一机器上加工的工序$\{i,j|i,j\in \{1,\cdots,N\}\wedge J_i \neq J_j\wedge M_i=M_j\}$,定义二进制变量$\{x_{i,j}|x_{i,j} \in \{0,1\}\wedge (i<j)\}$。

$x_{i,j}=1$意味着在这个变量对应的方案下,工序$i$比工序$j$更早加工;$x_{i,j}=0$意味着在这个变量对应的方案下,工序$i$比工序$j$更晚加工。

对应在析取图中,每个二进制变量$x_{i,j}$一对一对应于未确定方向的析取弧,因此当所有定义的二进制变量的值确定时,相应的调度方案也就确定了。我们将一个确定的调度方案对应的所有二进制变量取值的集合记作$\mathcal{x}$

然而,这样一个确定的调度方案并不一定是合理的,因为确定析取弧方向后的析取图可能是一个循环的有向图。我们不能从一个循环图中确定循环路径上工序的先后顺序。因此,相应的$\mathcal{x}$是一个不合理解。

解决不合理解的方法有两种:一种是在目标函数上附加对不合理解的惩罚项,这种方式会趋使生成的解朝着可行解的方向更新,但并不能杜绝不合理解,并且会使得目标函数更加复杂,增大优化的难度;另一种是在解的更新方式上作限制,使其总是在可行解的范围内迭代,这种方式
构建难度大,需要保证解的收敛性,往往需要进行额外的限制。